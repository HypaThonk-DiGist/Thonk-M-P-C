%Preamble
\documentclass{article}
\usepackage[utf8]{inputenc}
\usepackage{amsmath, amssymb, amsfonts}
\usepackage[dvipsnames*,svgnames]{xcolor}
\usepackage{tcolorbox, mdframed, hyperref, pgfplots, tikz, fontenc, xunicode, titlesec}
\setmathrm{\sffamily\textit{}}
\pgfplotsset{compat=1.17}
\tcbset{boxrule=0pt, width=5in, fonttitle=\sffamily\bfseries, sharp corners}
%------------------------------------------------------------
%------------------------------------------------------------
%THE DOCUMENT'S BEGINNING
\begin{document}
\sffamily
%------------------------------------------------------------
%------------------------------------------------------------
%SOME STUFF TO MAKE WORK EASY XD
\newenvironment{mathbox}[1][2]
{
  \vspace{1em}
  \begin{tcolorbox}[colback=yellow!50!white, colframe=yellow!60!red, title=#1]
  #
}
{
  \end{tcolorbox}
}
%THE MATH ONE
%------------------------------------------------------------
%------------------------------------------------------------
\newenvironment{phybox}[1][2]
{
  \vspace{1em}
  \begin{tcolorbox}[colback=blue!30!white, colframe=blue!80!white, title=#1]
  #
}
{
  \end{tcolorbox}
}
%THE PHYSICS ONE
%------------------------------------------------------------
%------------------------------------------------------------
\newenvironment{chembox}[1][2]
{
  \vspace{1em}
  \begin{tcolorbox}[colback=red!30!white, colframe=red!90!white, title=#1]
  #
}
{
  \end{tcolorbox}
}
%THE CHEMISTRY ONE
%------------------------------------------------------------
%------------------------------------------------------------
\newenvironment{biobox}[1][2]
{
  \vspace{1em}
  \begin{tcolorbox}[colback=green!30!white, colframe=ForestGreen!70!green, title=#1]
  #
}
{
  \end{tcolorbox}
}
%THE BIOLOGY ONE
%------------------------------------------------------------
%------------------------------------------------------------
\sffamily
\begin{align*}
    \textbf{\LARGE{One Learning a Day, One Subject at a Time}}
\end{align*}
%------------------------------------------------------------
\begin{align*}
\textbf{{\href{https://github.com/Psi25Omega/stuff-of-the-day}{\Large{Github Repository}}}}
\end{align*}
%------------------------------------------------------------
\begin{mathbox}
[\text{01-01-2021, A Simple Proof}]
Are there {infinitely many prime numbers}? If yes, how do we prove they exist?\\
Here's a simple proof.\\
{Assume} we have only $n$ prime numbers; $P_1, P_2, P_3, \dots P_n$.
\begin{align*}
\text{Let}~N = P_1 \cdot P_2 \cdot P_3 \dots P_n + 1
\end{align*}
$N$ {isn't divisible} by any of the primes $P_1, P_2, P_3, \dots P_n$\footnote{\sffamily{Prime numbers start from $2$ and $N$ is the LCM of all the prime numbers added to $1$.}}  which implies $N$'s {prime factorisation} is $N \times 1$. $N$ being prime {contradicts} our initial assumption.\\
Thus, there exist infinite primes. QED
\end{mathbox}
%------------------------------------------------------------
\begin{phybox}[\text{02-01-2021, \href{https://en.wikipedia.org/wiki/Quark}{Neutrons and Protons have Components too!}}]
Have you ever wondered whether {protons, neutrons and electrons; the constituents of an atom,} can be {divided further} into constituting components? The answer is {"Yes"}. A {quark} is an elementary particle and a fundamental constituent of matter. Quarks combine to form particles called hadrons. All commonly observable matter is composed of {up quarks, down quarks and electrons}. Quarks are never found existing individually, they can be found only composing {hadrons}, which include {baryons} (protons and neutrons) and {mesons}, or in {quark–gluon plasma}.
\end{phybox}
%------------------------------------------------------------
\begin{chembox}[\text{03-01-2021, \href{https://en.wikipedia.org/wiki/Ionic_bonding}{A Salty Bond}}]
The {Ionic bond} is a type of a {chemical bond} that is a result of the attraction {between oppositely charged particles} in ionic compounds like NaCl\footnote{\sffamily{More examples; KCl, CaCl_2}}. Ions are atoms (or a group of atoms) having a {net charge}. Atoms that {gain electrons} to become {stable} are called {anions} while those that {lose electrons} for the same are called {cations}. This {transfer of electrons} is known as {electrovalence}. Ionic bonds are {mostly} formed {between metals and non-metals}. In simpler words, an ionic bond is a result of the transfer of electrons from a metal (cation) to a non-metal (anion) in order for both atoms to attain stability.
\end{chembox}
%------------------------------------------------------------
\begin{biobox}[\text{04-01-2021, It's Hot. But it Helps}]
There are more than {10 million} cases of {Vitamin D deficiency} per year in India. Is Vitamin D really inaccessible? It is not! We can get Vitamin D from something as simple as {sunlight}. When our skin is exposed to sunlight, it makes Vitamin D (which, in reality, is {a hormone}) from {cholesterol}, a fat-rich and structural component of the animal cell membrane through the {energy} obtained from the sun’s {Ultraviolet-B (UVB)} rays and by the process of {Vitamin D synthesis}.\\
\textit{Work Hard, Play Harder!}
\end{biobox} 
%------------------------------------------------------------
\begin{mathbox}[\text{05-01-2021, \href{https://en.wikipedia.org/wiki/Logarithm}{Exponentiation^{-1}}}]
The {logarithm} function (denoted by $\log$) is the {inverse} function of {exponentiation}. It denotes the {exponent/power} to which the {'base'} has been raised to in a number. Here's an example;
\begin{align*}
    \text{Let } x = a^b. \text{ This implies} \log_a (x) = b
\end{align*}
The logarithm of $x$ to the base $a$ can be a {whole number, decimal number, irrational number} or even a {complex number} depending on its value. Often, log graph are used to make growth, forecast and even the number of cases during a pandemic. Logarithms involving complex numbers\footnote{\sffamily${\sqrt{-x}, \sqrt[3]{-x}, \dots}$} $(i)$ can be plotted on the real-complex plane.
\end{mathbox}
%------------------------------------------------------------
\begin{phybox}[\text{06-01-2020, \href{https://en.wikipedia.org/wiki/Uncertainty_principle}{Heisenberg's Life Questioning Inequality}}]
In quantum mechanics\footnote{\sffamily{Branch of Physics dedicated to observing the physical properties at the subatomic scale.}}, the uncertainty principle (also known as Heisenberg's uncertainty principle) is a variety of mathematical inequalities asserting a fundamental limit to the accuracy with which the values for certain pairs of physical quantities of a particle, such as position ($x$) and momentum ($p$) can be predicted from initial or known conditions. Heisenberg, in simple words stated that if you know velocity/momentum of the particle, you can't find it's position and vice versa. The equation stated in favour;
\begin{align*}
    \Delta x \times \Delta P \geq \frac{h}{4\pi}
\end{align*}
where  $h = 6.626 \times 10^-34$ (Planck's Constant).\\
The more precisely the position of some particle is determined, the less precisely its momentum/velocity can be predicted from the known initial conditions.
\end{phybox}
%------------------------------------------------------------
\begin{mathbox}[\text{08-01-2021, \href{https://en.wikipedia.org/wiki/Prime_number_theorem}{Prime Number Theorem}}]
The prime number theorem describes the asymptotic\footnote[1]{\sffamily{Asymptotic here means approximate in mathematical terms.}} distribution of prime numbers among positive integers. It formalizes the intuitive idea that primes become less common as they become larger. The prime number theorem addresses this by precisely quantifying the rate at their frequency decreases. The first breakthrough was the $\pi(n)$ function, which calculates the probability that a random integer less than or equal to $n$ is prime. It's defined as;
\begin{align*}
    \pi(n) \sim \frac{1}{\ln(n)}
\end{align*}
where $\ln (n)$ is the natural logarithm\footnote{$\log_e(n)$} of $n$.
\end{mathbox}
%------------------------------------------------------------
\begin{mathbox}[\text{Some Day, \href{https://en.wikipedia.org/wiki/Catalan\%27s_conjecture}{Catalan's Conjecture}}]
There exists only {one solution} where $x=3, a=2, y=2, b=3$ to the equation
\begin{align*} 
    x^a - y^b = 1
\end{align*} 
for {$a,b > 1$} and {$x,y > 0$}.
\end{mathbox}
%------------------------------------------------------------
\begin{phybox}[\text{Another Day, \href{https://en.wikipedia.org/wiki/Bohr_model#Origin}{Angular Momentum of an Electron}}]
The {angular momentum} $(L)$ of an {electron} in the $n^{th}$ orbit is given by 
\begin{align*} 
    L = \frac{nh}{2\pi} 
\end{align*} where $h$ is the {Planck's constant}. 
\end{phybox}
%------------------------------------------------------------
\begin{chembox}[\text{Some Other Day, \href{https://en.wikipedia.org/wiki/Gibbs_free_energy}{Gibbs Free Energy}}]
In thermodynamics, the {Gibbs Free Energy} $(G)$ (named after Josiah Willard Gibbs) is a {thermodynamic potential} that calculates the {maximum reversible work} performed by a thermodynamic system at a {constant temperature} $(T)$ and pressure{} $(P)$. It is given by 
\begin{align*} 
    \Delta G=\Delta H-T\Delta S 
\end{align*} where $S$ represents its {Entropy}, i.e. the measure of randomness. {S.I unit - Joules}.
\end{chembox}
%------------------------------------------------------------
%------------------------------------------------------------
\end{document}
