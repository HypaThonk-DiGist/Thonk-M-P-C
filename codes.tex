%Preamble%
\documentclass{article}
\usepackage[utf8]{inputenc}
\usepackage{amsmath, amssymb, amsfonts}
\usepackage{pgfplots, tikz}
\usepackage{tcolorbox, mdframed, hyperref}
\pgfplotsset{compat=1.17}
\tcbset{boxrule=0pt, width=5in, fonttitle=\sffamily\bfseries, sharp corners}
%
\begin{document}
\begin{align*}
    \sffamily\LARGE\textbf{Stuff of the Day - Compiled Version}
\end{align*}
\sffamily
\begin{align*}
\textbf{\href{https://github.com/Psi25Omega/stuff-of-the-day}{Github Repository}}
\end{align*}
%
%Code%
%
\begin{tcolorbox}[title=\text{30-12-2020, Catalan's Conjecture}]
There exists only one solution where $x=3, a-2, y=2, b=3$ to the equation 
\begin{align*} 
    x^a - y^b = 1 
\end{align*} 
for $a,b > 1$ and $x,y > 0$.
\end{tcolorbox}
%
\begin{tcolorbox}[title=\text{31-12-2020, Angular Momentum of an Electron}]
The angular momentum $(L)$ of an electron in the $n^{th}$ orbit is given by 
\begin{align*} 
    L = \frac{nh}{2\pi} 
\end{align*} where $h$ is the Planck's constant. 
\begin{flushright} \textit{Niels Bohr} \end{flushright}
\end{tcolorbox}
%
\begin{tcolorbox}[title=\text{01-01-2021, Gibbs Free Energy}]
In thermodynamics, the Gibbs Free Energy $(G)$ (named after Josiah Willard Gibbs) is a thermodynamic potential that calculates the maximum reversible work  performed by a thermodynamic system at a constant temperature $(T)$ and pressure $(P)$. It is given by 
\begin{align*} 
    \Delta G=\Delta H-T\Delta S 
\end{align*} where $S$ represents its Entropy, i.e. the measure of randomness. S.I unit - Joules.
\end{tcolorbox}
%
\end{document}
