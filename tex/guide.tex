\section{Preface}
\subsection{A Note to the Reader}
\subsubsection{Behind the Idea - A Short Story}
This book is the result of a lot of contemplation. Psi25Omega was thinking about stuff he would like to learn and read about each day when he thought, "Why not take notes and make it a document? Why not 'LaTeX' it? It certainly would look good as well!". And then, he played with the different styles and options available for a few days and had a basic draft of the document. It was after this that he invited NSPKN6506 to the party after both of them found the project interesting to work on and helpful to people as well. And they have come this far, from struggling with multiple errors due to bad code (Yeah, they were not so good at LaTeX back then XD) and filling up the whole of January with content (Again, if you read it on par with the expectations; i.e. one thing a day) to the most recent draft you are gonna read. If you feel they did well, do try saying a "Hi" to them in school or on Discord or anywhere you meet them (You can discuss more about these things). They would love to clarify any doubts or clarifications from the book.
\subsubsection{How you can start reading!}
So, now you have finished the boring Introduction and have come to the part where you get some advice on how you can start reading the book. The thing is, there is no defined way in which you should read this. We recommend reading one "Box" a day as you can later search it up and gather more information going deeper into the topic. The content is a basic gist of the topic and tries kindle your interest to go to a browser and type in the topic on the search bar.
\subsubsection{Final thoughts}
Here are some final thoughts and reflections on the book. Please forgive us for any mistakes in the book and we'll try to correct them whenever you guys submit a pull request or we notice it. The book is of a higher level than standard Science and Math curriculum in India (9th and 10th grades). But don't worry, you can still understand most (95\%) of the stuff (You can always use the internet to know about the things you don't understand). From another country's perspective, the book may be too trivial or tough, again depending on your curriculum's rigour. Overall, it's just another book that lists facts and theorems, but hopefully, in an inquisitive and interesting way.
\subsection{Notations}
\subsubsection{The Box!}
Here is an example of the typical "box" we have used in the book. We also colour code the boxes to represent the different subjects the things cover;
\begin{center}
\begin{tabular}{|c|c|}
\hline
Subject & Colour\\
\hline
Math & Yellow\\
Physics & Blue\\
Chemistry & Red\\
\hline
\end{tabular}
\end{center}
\begin{guidebox}{\text{The Title (Sometimes involves Bad Puns)}}
{This so-called "box" usually contains the content and sometimes includes images, graphs or diagrams to make things interesting and clear. At times, we also have included footnotes\footnote{They are for clarifications that aren't really obvious but aren't significant at the same time.} to be more reader-friendly.}
\end{guidebox}
\subsubsection{Contributing}
If you want to contribute to the project, feel free to fork our Github Repository and submit pull requests or get in touch with one of the authors.
\subsubsection{Github}
This is a link to our Github Repository where we host most of the Static files and LaTeX code.
\begin{flushleft}
\color{blue}
\textbf{\href{https://github.com/Psi25Omega/Thonk-M-P-C}{Github, Thonk}}
\end{flushleft}
%------------------------------------------------------------
\newpage
%------------------------------------------------------------

