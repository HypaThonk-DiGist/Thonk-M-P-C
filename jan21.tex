%Preamble
\documentclass{article}
\usepackage[utf8]{inputenc}
\usepackage{amsmath, amssymb, amsfonts}
\usepackage{tcolorbox, mdframed, hyperref, pgfplots, tikz, xcolor}
\pgfplotsset{compat=1.17}
\tcbset{boxrule=0pt, width=5in, fonttitle=\sffamily\bfseries, sharp corners}
%------------------------------------------------------------
%------------------------------------------------------------
%THE DOCUMENT'S BEGINNING
\begin{document}
\sffamily
%------------------------------------------------------------
%------------------------------------------------------------
%SOME STUFF TO MAKE WORK EASY XD
\newenvironment{mathbox}[1][2]
{
  \vspace{1em}
  \begin{tcolorbox}[colback=yellow!50!white, colframe=yellow!60!red, title=#1]
  #
}
{
  \end{tcolorbox}
}
%THE MATH ONE
%------------------------------------------------------------
%------------------------------------------------------------
\newenvironment{phybox}[1][2]
{
  \vspace{1em}
  \begin{tcolorbox}[colback=blue!30!white, colframe=blue!80!white, title=#1]
  #
}
{
  \end{tcolorbox}
}
%THE PHYSICS ONE
%------------------------------------------------------------
%------------------------------------------------------------
\newenvironment{chembox}[1][2]
{
  \vspace{1em}
  \begin{tcolorbox}[colback=red!30!white, colframe=red!90!white, title=#1]
  #
}
{
  \end{tcolorbox}
}
%THE CHEMISTRY ONE
%------------------------------------------------------------
%------------------------------------------------------------
\newenvironment{biobox}[1][2]
{
  \vspace{1em}
  \begin{tcolorbox}[colback=green!30!white, colframe=green!90!black, title=#1]
  #
}
{
  \end{tcolorbox}
}
%THE BIOLOGY ONE
%------------------------------------------------------------
%------------------------------------------------------------
%THE REAL DEAL STARTS HERE
\begin{align*}
    \LARGE\textbf{One Learning a Day, One Subject at a Time}
\end{align*}
%------------------------------------------------------------
\begin{align*}
\textbf{\href{https://github.com/Psi25Omega/stuff-of-the-day}{\Large{Github Repository}}}
\end{align*}
%------------------------------------------------------------
%------------------------------------------------------------
%THAT WAS A JOKE, IT STARTS HERE
\begin{mathbox}
[\text{01-01-2021, A Simple Proof}]
Are there \textbf{infinite primes}? If yes, how do we prove they exist?\\
Here's a simple proof.\\
Assume we have only $n$ prime numbers; $P_1, P_2, P_3, \dots P_n$.\\
$\text{Let}~N = P_1 \cdot P_2 \cdot P_3 \dots P_n + 1$\\
$N$ isn't divisible by any of the primes $P_1, P_2, P_3, \dots P_n$  which implies $N$'s \textbf{prime factorisation} is $N \times 1$. $N$ being prime \textbf{contradicts} our initial assumption.\\
Thus, there exist infinite primes.
\end{mathbox}
%------------------------------------------------------------
\begin{phybox}[\text{02-01-2021, \href{https://en.wikipedia.org/wiki/Quark}{Neutrons and Protons have Components too!}}]
Have you even wondered whether \textbf{protons, neutrons and electrons; the constituents of an atom,} can be \textbf{divided further} into constituting components? The answer is \textbf{"Yes"}. A \textbf{quark} is an elementary particle and a fundamental constituent of matter. Quarks combine to form particles called hadrons. All commonly observable matter is composed of \textbf{up quarks, down quarks and electrons}. Quarks are never found existing individually, they can be found only composing \textbf{hadrons}, which include \textbf{baryons} (protons and neutrons) and \textbf{mesons}, or in \textbf{quark–gluon plasma}.
\end{phybox}
%------------------------------------------------------------
\begin{chembox}[\text{03-01-2021, \href{https://en.wikipedia.org/wiki/Ionic_bonding}{A Salty Bond}}]
The \textbf{Ionic bond} is a type of a \textbf{chemical bond} that is a result of the attraction \textbf{between oppositely charged particles} in ionic compounds like NaCl. Ions are atoms (or a group of atoms) having a \textbf{net charge}. Atoms that \textbf{gain electrons} to become \textbf{stable} are called \textbf{anions} while those that \textbf{lose electrons} for the same are called \textbf{cations}. This \textbf{transfer of electrons} is known as \textbf{electrovalence}. Ionic bonds are \textbf{mostly} formed \textbf{between metals and non-metals}. In simpler words, an ionic bond is a result of the transfer of electrons from a metal (cation) to a non-metal (anion) in order for both atoms to attain stability.
\end{chembox}
%------------------------------------------------------------
\begin{biobox}[\text{04-01-2021, It's Hot. But it Helps}]
There are more than \textbf{10 million} cases of \textbf{Vitamin D deficiency} per year in India. Is Vitamin D really inaccessible? It is not! We can get Vitamin D from something as simple as \textbf{sunlight}. When our skin is exposed to sunlight, it makes Vitamin D (which, in reality, is \textbf{a hormone}) from \textbf{cholesterol}, a fat-rich and structural component of the animal cell membrane through the \textbf{energy} obtained from the sun’s \textbf{Ultraviolet-B (UVB)} rays and by the process of \textbf{Vitamin D synthesis}.\\
\textit{Work Hard, Play Harder!} :')
\end{biobox} 
%------------------------------------------------------------
\begin{mathbox}[\text{Some Day, \href{https://en.wikipedia.org/wiki/Catalan\%27s_conjecture}{Catalan's Conjecture}}]
There exists only \textbf{one solution} where $x=3, a=2, y=2, b=3$ to the equation
\begin{align*} 
    \textbf{$x^a - y^b = 1$} 
\end{align*} 
for \textbf{$a,b > 1$} and \textbf{$x,y > 0$}.
\end{mathbox}
%------------------------------------------------------------
\begin{phybox}[\text{Another Day, \href{https://en.wikipedia.org/wiki/Bohr_model#Origin}{Angular Momentum of an Electron}}]
The \textbf{angular momentum} $(L)$ of an \textbf{electron} in the $n^{th}$ orbit is given by 
\begin{align*} 
    L = \frac{nh}{2\pi} 
\end{align*} where $h$ is the \textbf{Planck's constant}. 
\end{phybox}
%------------------------------------------------------------
\begin{chembox}[\text{Some Other Day, \href{https://en.wikipedia.org/wiki/Gibbs_free_energy}{Gibbs Free Energy}}]
In thermodynamics, the \textbf{Gibbs Free Energy} $(G)$ (named after Josiah Willard Gibbs) is a \textbf{thermodynamic potential} that calculates the \textbf{maximum reversible work} performed by a thermodynamic system at a \textbf{constant temperature} $(T)$ and pressure\textbf{} $(P)$. It is given by 
\begin{align*} 
    \Delta G=\Delta H-T\Delta S 
\end{align*} where $S$ represents its \textbf{Entropy}, i.e. the measure of randomness. \textbf{S.I unit - Joules}.
\end{chembox}
%------------------------------------------------------------
%------------------------------------------------------------
\end{document}
