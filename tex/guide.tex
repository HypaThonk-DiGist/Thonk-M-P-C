\section{Preface}
\subsection{A Note to the Reader}
\subsubsection{Behind the Idea}
\subsubsection{How you can start reading!}
\subsubsection{Final thoughts}
\subsection{Notations}
Here is a list of notations used in the book.
\subsubsection{The Box!}
This is an example of the "box" where we have our title and the content. We colour code it to represent the different subjects the things cover;
\begin{flushleft}
\begin{tabular}{|c|c|}
\hline
Subject & Colour\\
\hline
Math & Yellow\\
Physics & Blue\\
Chemistry & Red\\
\hline
\end{tabular}
\end{flushleft}
\begin{guidebox}{\text{The Title (Sometimes involves Bad Puns)}}
{This are usually contains the content and sometimes includes images, graphs, diagrams to make things interesting and clear. At times, we also have included footnotes\footnote{They are for clarifications that are not so obvious but aren't significant.} to be more reader-friendly.}
\end{guidebox}
\subsection{Contributing}
If you want to contribute to the project, feel free to fork our Github Repository and submit pull requests or get in touch with one of the authors.
\subsubsection{Github}
This is a link to our Github Repository where we host most of the Static Files and we love Open-Source (Wait, who does not?), so we consider this a step towards accessible knowledge.\\
\textbf{\href{https://github.com/Psi25Omega/Thonk-M-P-C}{Repo Link}}
